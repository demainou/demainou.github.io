\documentclass[]{book}
\usepackage{lmodern}
\usepackage{amssymb,amsmath}
\usepackage{ifxetex,ifluatex}
\usepackage{fixltx2e} % provides \textsubscript
\ifnum 0\ifxetex 1\fi\ifluatex 1\fi=0 % if pdftex
  \usepackage[T1]{fontenc}
  \usepackage[utf8]{inputenc}
\else % if luatex or xelatex
  \ifxetex
    \usepackage{mathspec}
  \else
    \usepackage{fontspec}
  \fi
  \defaultfontfeatures{Ligatures=TeX,Scale=MatchLowercase}
\fi
% use upquote if available, for straight quotes in verbatim environments
\IfFileExists{upquote.sty}{\usepackage{upquote}}{}
% use microtype if available
\IfFileExists{microtype.sty}{%
\usepackage{microtype}
\UseMicrotypeSet[protrusion]{basicmath} % disable protrusion for tt fonts
}{}
\usepackage[margin=1in]{geometry}
\usepackage{hyperref}
\hypersetup{unicode=true,
            pdftitle={DATA SCIENCE FOR A CITY - SOUTH AFRICAN LOCAL GOVERNMENT},
            pdfauthor={Delyno du Toit},
            pdfborder={0 0 0},
            breaklinks=true}
\urlstyle{same}  % don't use monospace font for urls
\usepackage{natbib}
\bibliographystyle{apalike}
\usepackage{longtable,booktabs}
\usepackage{graphicx,grffile}
\makeatletter
\def\maxwidth{\ifdim\Gin@nat@width>\linewidth\linewidth\else\Gin@nat@width\fi}
\def\maxheight{\ifdim\Gin@nat@height>\textheight\textheight\else\Gin@nat@height\fi}
\makeatother
% Scale images if necessary, so that they will not overflow the page
% margins by default, and it is still possible to overwrite the defaults
% using explicit options in \includegraphics[width, height, ...]{}
\setkeys{Gin}{width=\maxwidth,height=\maxheight,keepaspectratio}
\IfFileExists{parskip.sty}{%
\usepackage{parskip}
}{% else
\setlength{\parindent}{0pt}
\setlength{\parskip}{6pt plus 2pt minus 1pt}
}
\setlength{\emergencystretch}{3em}  % prevent overfull lines
\providecommand{\tightlist}{%
  \setlength{\itemsep}{0pt}\setlength{\parskip}{0pt}}
\setcounter{secnumdepth}{5}
% Redefines (sub)paragraphs to behave more like sections
\ifx\paragraph\undefined\else
\let\oldparagraph\paragraph
\renewcommand{\paragraph}[1]{\oldparagraph{#1}\mbox{}}
\fi
\ifx\subparagraph\undefined\else
\let\oldsubparagraph\subparagraph
\renewcommand{\subparagraph}[1]{\oldsubparagraph{#1}\mbox{}}
\fi

%%% Use protect on footnotes to avoid problems with footnotes in titles
\let\rmarkdownfootnote\footnote%
\def\footnote{\protect\rmarkdownfootnote}

%%% Change title format to be more compact
\usepackage{titling}

% Create subtitle command for use in maketitle
\newcommand{\subtitle}[1]{
  \posttitle{
    \begin{center}\large#1\end{center}
    }
}

\setlength{\droptitle}{-2em}
  \title{DATA SCIENCE FOR A CITY - SOUTH AFRICAN LOCAL GOVERNMENT}
  \pretitle{\vspace{\droptitle}\centering\huge}
  \posttitle{\par}
  \author{Delyno du Toit}
  \preauthor{\centering\large\emph}
  \postauthor{\par}
  \predate{\centering\large\emph}
  \postdate{\par}
  \date{2018-04-28}

\usepackage{booktabs}
\usepackage{amsthm}
\makeatletter
\def\thm@space@setup{%
  \thm@preskip=8pt plus 2pt minus 4pt
  \thm@postskip=\thm@preskip
}
\makeatother

\usepackage{amsthm}
\newtheorem{theorem}{Theorem}[chapter]
\newtheorem{lemma}{Lemma}[chapter]
\theoremstyle{definition}
\newtheorem{definition}{Definition}[chapter]
\newtheorem{corollary}{Corollary}[chapter]
\newtheorem{proposition}{Proposition}[chapter]
\theoremstyle{definition}
\newtheorem{example}{Example}[chapter]
\theoremstyle{definition}
\newtheorem{exercise}{Exercise}[chapter]
\theoremstyle{remark}
\newtheorem*{remark}{Remark}
\newtheorem*{solution}{Solution}
\begin{document}
\maketitle

{
\setcounter{tocdepth}{1}
\tableofcontents
}
\chapter{Title page}\label{title-page}

Declaration English - Abstract (max. 500 words) Afrikaans - Opsomming
(max. 500 words) Acknowledgements Table of Contents List of Figures List
of Tables Main Body -- e.g.~Chapter 1, Chapter 2, etc. Reference List
Addenda (e.g.~Addendum A, Addendum B, etc.)

\chapter{INTRODUCTION}\label{introduction}

\section{Background}\label{background}

The public sector, more than the private sector, is confined by laws,
bylaws, policies, and a risk-averse mindset. Layers of bureaucracy,
inflexible rule applications, political agendas, multiple agency
involvement curb progressive breakthroughs. Financial stresses are
adding to local governments restrictions making them even less
responsive. All the constraints limit local government to focus on
outpus rather than the bigger picture and solutions. Officials get
measured on how many houses are build, how quickly are potholes filled,
how much lenghts of road is build rather than actual reductions in
homelessness, reduction in potholes, efficient public transportation.
Local government is protected against any abuse of discretion through an
orientation toward compliance instead of the impact on society.
Complexity and rule-driven accountability affect the way officials
operates and regulate. Officials who've been with government for long
periods, have slumped into a culture of dealing with problems by
applying older bureaucratic structures to enforce compliance, resulting
in inefficient utilisation of resources.

Yet local government has the means to completely reverse this trend
toward despair. One of the opportunities presenting itself is in the
form of data modernisation. With the digital tools available today,
officials can revolutionize local government, making it more responsive,
transparent, and cost-effective than it has ever been.

Local government is structured into compartmentalized units, with each
having a unique mandate. The South African city used for this
dissertation consist of 11 directorates. A directorate can be thought of
as a company within the amalgamation of the city:

\begin{enumerate}
\def\labelenumi{\arabic{enumi}.}
\tightlist
\item
  Area-Based Service Delivery
\item
  Assets and Facilities Management
\item
  Corporate Services
\item
  Directorate of the Mayor
\item
  Energy
\item
  Finance
\item
  Informal Settlements, Water and Waste Services
\item
  Office of the City Manager
\item
  Safety and Security
\item
  Social Services
\item
  Transport \& Urban Development Authority
\end{enumerate}

Each of the directorates have a specific mandate that they need to
realize. This leads to directorates working in isolation and duplicating
efforts and investments when managing their area. Although each
directorate is unique, they must all work together towards a common
purpose of providing a safe-, caring-, inclusive-, opportunity- and
well-run city . Big strides have been made to create a transversal
approach to solving problems and to break down the silo way of working.

Daily, decisions get taken that have an impact on the city's operations,
strategies, urban plans and finances. The City leadership, is dependent
on data to inform their decision-making. The problem is that the data
used to inform decision making resides on multiple systems and with
multiple people. The systems don't talk to each other and to do
performance reporting is cumbersome, let alone analyses. Accessing the
data is resource intensive and time consuming. Once the data is collated
more work need to be performed on it before presenting it for decision
making. The data presented is open for interpretation and limited or
non-existent in analysis to provide insights to improve performance and
get the maximum effect with limited resources. Due to the manual
interventions to synthesis the data errors creep in and more concerning
different figures are being reported on for the same information.

When it comes to decision-making the city systems is restricted to
reporting on the pass and by function, e.g.~service requests, revenue,
etc. The main system used for this is the ERP system. The reports are
reflective and present data in a static format, i.e.~not visual or
interactive or highlighting exceptions, trends or providing insight from
the data. Limited analysis is done on the data on a functional level,
and on a transversal level even less. The non-transactional data is used
haphazardly during the decision-making process, i.e.~survey's, images,
videos, sensors, external data from national and provincial government
and private sector, etc.

Within the City the term ``data-driven decision-making'' means reporting
key performance metrics based on historical data and using these metrics
to support and justify business decisions. This is a good start but data
volume, variety and velocity is ever-increasing. To capitalize on
opportunities that can be identified, data and analytics should take on
a more active and dynamic role in powering the activities of the entire
organization, not just reflecting where it's been.

Given all the data there is is still an omission of data-driven
decisions. Decisions should no longer be left to gut instinct. Instead,
decisions and actions should be based in facts, and those facts also
fuel algorithms that predict optimal outcomes. Although it's taken a
while to take root, leading enterprises are finally embracing this
perspective. The city's IT systems main purpose is to automate
processes. Data is stored, then analyzed, often as an afterthought, to
assess what had already happened. That passive approach need to give way
to a more proactive, engaged model, where systems are architected and
built around analytics.

The box below provide a good summary of the bottlenecks faced within the
city.

\begin{verbatim}
In New York City, Michael Bloomberg took office as mayor (January 1, 2002 – December 31, 2013)
after long years of experience in the use of data, and he created a metrics-driven mayoralty.
Agencies agreed to cooperate to set up his proposed data analytics center and other interagency
data initiatives. Yet almost all of them soon asserted legal, technical, and operational obstacles
to full participation. Budget experts also pushed back, worried about costs. Lawyers cited vast
numbers of rules (most from the federal government) that prohibited sharing of data.  Within each
city agency, its chief information officer would explain how only he or she could manage the
complex legacy databases of that unit. Despite his mandate, his commitment to data, and a raft
of first-rate appointees, Bloomberg would not have succeeded in making New York City a leader in
data-driven government had he not pushed hard from the top for change. The lesson here is a bit
paradoxical: How can leaders at the top of a hierarchy create the conditions that will replace
that hierarchy with a far more open and fluid system?
\end{verbatim}

\citep{responsiveCity} The Responsive City

Unleashing the power of data and analytics will bring about the
disintegration of the age of bureaucracy, allowing government to move
from a compliance model to a problem-solving one that truly values the
intelligence and dedication of its employees and the imagination and
civic spirit of its citizens.

\textbf{Data Science}

\textbf{City Dashboards}

\textbf{Cities Current Data Science Environment}

\section{Problem Statement}\label{problem-statement}

How can a South African city use the vast amount of data in their
posession to make better decsions by applying data science. The
dissertation will address this question by applying data science
techniques to three use cases.

\section{Definitions, assumptions,
limitations}\label{definitions-assumptions-limitations}

\section{Brief Chapter overviews}\label{brief-chapter-overviews}

The dissertation consist of five chapters.

\begin{itemize}
\item
  In Chapter 1, a compelling case is made for the application of data
  science within local government.
\item
  Chapter 2 serves as the foundation on which the study is built and as
  a basis for discussing results and interpretations. This chapter
  situates the study in the context of previous research and scholarly
  material pertaining to data science within local government or
  government, presents a critical synthesis of empirical literature
  according to relevant themes or variables, justifies how the study
  addresses a gap or problem in the literature, and outlines the
  theoretical or conceptual framework of the study.
\item
  Chapter 3 provides a rationale for the methodology followed, describes
  the research setting and sample, and describes data collection and
  analysis methods. The chapter provides a detailed description of all
  aspects of the design and procedures of the study.
\item
  Chapter 4 present a factual reporting of the study results is
  presented. The chapter offers the researcher an opportunity to reflect
  on the study's findings, and the practical and theoretical
  implications thereof.
\item
  Chapter 5 presents a set of concluding statements and recommendations.
  ``Knowing what I now know, what conclusion can I draw?'' This may
  include implications for practice as well as implications for future
  research. Findings are integrated with the theory employed in the
  first chapter and the body of knowledge presented in the second
  chapter. The chapter ends with a cogent conclusion summarizing the
  importance of the study findings.
\end{itemize}

\chapter{LITERATURE REVIEW}\label{literature-review}

\section{Introduction}\label{introduction-1}

\section{Broad context theory base}\label{broad-context-theory-base}

\section{Detailed (works organised by topic or
idea)}\label{detailed-works-organised-by-topic-or-idea}

\section{Conclusion}\label{conclusion}

\chapter{METHOD}\label{method}

\section{Introduction}\label{introduction-2}

\section{Research Design}\label{research-design}

\section{Research Instruments}\label{research-instruments}

\section{Data}\label{data}

\section{Analyses}\label{analyses}

\section{Limitations}\label{limitations}

\section{Ethics}\label{ethics}

\section{Conclusion}\label{conclusion-1}

\chapter{FINDINGS AND ANALYSES}\label{findings-and-analyses}

\section{Use Case 1: Predictive Maintenance of a city's Electricity
Assets}\label{use-case-1-predictive-maintenance-of-a-citys-electricity-assets}

\subsection{introduction}\label{introduction-3}

\subsection{sections}\label{sections}

\subsection{sub conclusions}\label{sub-conclusions}

\section{Use Case 2: Image Classification of Informal
Dwellings}\label{use-case-2-image-classification-of-informal-dwellings}

\subsection{introduction}\label{introduction-4}

\subsection{sections}\label{sections-1}

\subsection{sub conclusions}\label{sub-conclusions-1}

\section{Use Case 3: Supervised- and Unsupervised Learning applied to a
city's Service
Notification}\label{use-case-3-supervised--and-unsupervised-learning-applied-to-a-citys-service-notification}

\subsection{introduction}\label{introduction-5}

\subsection{sections}\label{sections-2}

\subsection{sub conclusions}\label{sub-conclusions-2}

\chapter{CONCLUSION}\label{conclusion-2}

\section{Summary of Findings}\label{summary-of-findings}

\section{Conclusions}\label{conclusions}

\section{Summary of Contributions}\label{summary-of-contributions}

\section{Future Research}\label{future-research}

\bibliography{book.bib}


\end{document}
